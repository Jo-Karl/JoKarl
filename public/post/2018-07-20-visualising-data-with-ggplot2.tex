\documentclass[]{article}
\usepackage{lmodern}
\usepackage{amssymb,amsmath}
\usepackage{ifxetex,ifluatex}
\usepackage{fixltx2e} % provides \textsubscript
\ifnum 0\ifxetex 1\fi\ifluatex 1\fi=0 % if pdftex
  \usepackage[T1]{fontenc}
  \usepackage[utf8]{inputenc}
\else % if luatex or xelatex
  \ifxetex
    \usepackage{mathspec}
  \else
    \usepackage{fontspec}
  \fi
  \defaultfontfeatures{Ligatures=TeX,Scale=MatchLowercase}
\fi
% use upquote if available, for straight quotes in verbatim environments
\IfFileExists{upquote.sty}{\usepackage{upquote}}{}
% use microtype if available
\IfFileExists{microtype.sty}{%
\usepackage{microtype}
\UseMicrotypeSet[protrusion]{basicmath} % disable protrusion for tt fonts
}{}
\usepackage[margin=1in]{geometry}
\usepackage{hyperref}
\hypersetup{unicode=true,
            pdftitle={Multi-dimensional scaling in R},
            pdfauthor={Johannes Karl},
            pdfborder={0 0 0},
            breaklinks=true}
\urlstyle{same}  % don't use monospace font for urls
\usepackage{color}
\usepackage{fancyvrb}
\newcommand{\VerbBar}{|}
\newcommand{\VERB}{\Verb[commandchars=\\\{\}]}
\DefineVerbatimEnvironment{Highlighting}{Verbatim}{commandchars=\\\{\}}
% Add ',fontsize=\small' for more characters per line
\usepackage{framed}
\definecolor{shadecolor}{RGB}{248,248,248}
\newenvironment{Shaded}{\begin{snugshade}}{\end{snugshade}}
\newcommand{\KeywordTok}[1]{\textcolor[rgb]{0.13,0.29,0.53}{\textbf{#1}}}
\newcommand{\DataTypeTok}[1]{\textcolor[rgb]{0.13,0.29,0.53}{#1}}
\newcommand{\DecValTok}[1]{\textcolor[rgb]{0.00,0.00,0.81}{#1}}
\newcommand{\BaseNTok}[1]{\textcolor[rgb]{0.00,0.00,0.81}{#1}}
\newcommand{\FloatTok}[1]{\textcolor[rgb]{0.00,0.00,0.81}{#1}}
\newcommand{\ConstantTok}[1]{\textcolor[rgb]{0.00,0.00,0.00}{#1}}
\newcommand{\CharTok}[1]{\textcolor[rgb]{0.31,0.60,0.02}{#1}}
\newcommand{\SpecialCharTok}[1]{\textcolor[rgb]{0.00,0.00,0.00}{#1}}
\newcommand{\StringTok}[1]{\textcolor[rgb]{0.31,0.60,0.02}{#1}}
\newcommand{\VerbatimStringTok}[1]{\textcolor[rgb]{0.31,0.60,0.02}{#1}}
\newcommand{\SpecialStringTok}[1]{\textcolor[rgb]{0.31,0.60,0.02}{#1}}
\newcommand{\ImportTok}[1]{#1}
\newcommand{\CommentTok}[1]{\textcolor[rgb]{0.56,0.35,0.01}{\textit{#1}}}
\newcommand{\DocumentationTok}[1]{\textcolor[rgb]{0.56,0.35,0.01}{\textbf{\textit{#1}}}}
\newcommand{\AnnotationTok}[1]{\textcolor[rgb]{0.56,0.35,0.01}{\textbf{\textit{#1}}}}
\newcommand{\CommentVarTok}[1]{\textcolor[rgb]{0.56,0.35,0.01}{\textbf{\textit{#1}}}}
\newcommand{\OtherTok}[1]{\textcolor[rgb]{0.56,0.35,0.01}{#1}}
\newcommand{\FunctionTok}[1]{\textcolor[rgb]{0.00,0.00,0.00}{#1}}
\newcommand{\VariableTok}[1]{\textcolor[rgb]{0.00,0.00,0.00}{#1}}
\newcommand{\ControlFlowTok}[1]{\textcolor[rgb]{0.13,0.29,0.53}{\textbf{#1}}}
\newcommand{\OperatorTok}[1]{\textcolor[rgb]{0.81,0.36,0.00}{\textbf{#1}}}
\newcommand{\BuiltInTok}[1]{#1}
\newcommand{\ExtensionTok}[1]{#1}
\newcommand{\PreprocessorTok}[1]{\textcolor[rgb]{0.56,0.35,0.01}{\textit{#1}}}
\newcommand{\AttributeTok}[1]{\textcolor[rgb]{0.77,0.63,0.00}{#1}}
\newcommand{\RegionMarkerTok}[1]{#1}
\newcommand{\InformationTok}[1]{\textcolor[rgb]{0.56,0.35,0.01}{\textbf{\textit{#1}}}}
\newcommand{\WarningTok}[1]{\textcolor[rgb]{0.56,0.35,0.01}{\textbf{\textit{#1}}}}
\newcommand{\AlertTok}[1]{\textcolor[rgb]{0.94,0.16,0.16}{#1}}
\newcommand{\ErrorTok}[1]{\textcolor[rgb]{0.64,0.00,0.00}{\textbf{#1}}}
\newcommand{\NormalTok}[1]{#1}
\usepackage{graphicx,grffile}
\makeatletter
\def\maxwidth{\ifdim\Gin@nat@width>\linewidth\linewidth\else\Gin@nat@width\fi}
\def\maxheight{\ifdim\Gin@nat@height>\textheight\textheight\else\Gin@nat@height\fi}
\makeatother
% Scale images if necessary, so that they will not overflow the page
% margins by default, and it is still possible to overwrite the defaults
% using explicit options in \includegraphics[width, height, ...]{}
\setkeys{Gin}{width=\maxwidth,height=\maxheight,keepaspectratio}
\IfFileExists{parskip.sty}{%
\usepackage{parskip}
}{% else
\setlength{\parindent}{0pt}
\setlength{\parskip}{6pt plus 2pt minus 1pt}
}
\setlength{\emergencystretch}{3em}  % prevent overfull lines
\providecommand{\tightlist}{%
  \setlength{\itemsep}{0pt}\setlength{\parskip}{0pt}}
\setcounter{secnumdepth}{0}
% Redefines (sub)paragraphs to behave more like sections
\ifx\paragraph\undefined\else
\let\oldparagraph\paragraph
\renewcommand{\paragraph}[1]{\oldparagraph{#1}\mbox{}}
\fi
\ifx\subparagraph\undefined\else
\let\oldsubparagraph\subparagraph
\renewcommand{\subparagraph}[1]{\oldsubparagraph{#1}\mbox{}}
\fi

%%% Use protect on footnotes to avoid problems with footnotes in titles
\let\rmarkdownfootnote\footnote%
\def\footnote{\protect\rmarkdownfootnote}

%%% Change title format to be more compact
\usepackage{titling}

% Create subtitle command for use in maketitle
\newcommand{\subtitle}[1]{
  \posttitle{
    \begin{center}\large#1\end{center}
    }
}

\setlength{\droptitle}{-2em}

  \title{Multi-dimensional scaling in R}
    \pretitle{\vspace{\droptitle}\centering\huge}
  \posttitle{\par}
    \author{Johannes Karl}
    \preauthor{\centering\large\emph}
  \postauthor{\par}
      \predate{\centering\large\emph}
  \postdate{\par}
    \date{2018-07-20}


\begin{document}
\maketitle

\section{MDS basics}\label{mds-basics}

This post focuses on MDS (Also sometimes called smallest space analysis
(Guttman, 1968) or multidimensional similarity structure analysis (Borg
\& Lingoes, 1987)) as exploratory tool and a large amount of the
information is lifted from Borg and Gronen (2005). Exploratory MDS might
seem at first glance like tea-leaf reading, but over the years several
rules have been developed that aid in determining structure from the MDS
plots.

In the case I will talk about during this post we were interested in
discovering how different measures of mindfulness, personality, and
approach/ avoidance motivation are related to each other. To this extent
participants answered statements about themselves on Likert-scales
related to these concepts. An important concept in MDS is distance. The
idea behind distance is that individuals reproduce mental distances
between concepts when asked about dissimilarities between the concepts.
To cite directly form Borg and Gronen (2005): " The most common approach
is to hypothesize that a person, when asked about the dissimilarity of
pairs of objects from a set of objects, acts as if he or she computes a
distance in his or her ``psychological space'' of these objects."

In our case participants were not asked to rate dissimilarities between
the measures, but rather rate themselves on these measures.
Nevertheless, if we think about it these self-ratings can also represent
psychological distances. A participant that scores high on avoidance
behaviour might also score high on anxiety, but score low on approach
behaviour. These differences can be represented as psychological
distances with avoidance behaviour and anxiety being in close proximity
and approach behaviour being distal from those two concepts. \#\#
Correlations and MDS Data obtained from answers to Likert scales can not
directly be considered as distances, but the correlation coefficients
between multiple columns can. Correlations are appropriate for MDS.

\subsubsection{Obtaining a distance matrix from a correlation
matrix.}\label{obtaining-a-distance-matrix-from-a-correlation-matrix.}

The \emph{psych} package provides a handy function (\emph{cor2dist}) to
transform a correlation matrix into a distance matrix. In the background
this simple function is performed: \[\sqrt{2 * (1 - x)}\]

Pipeline provide a handy way of reducing the number of objects stored in
the work space if they are not used in later analysis. In my current
case I used the different measures of interested (Mindfulness facets,
personality, etc.) and labeled the facets. The first line produces the
correlation between the variables, the second line selects only the
correlation coefficients from the output, and the second line converts
them into a distance matrix which is returned as an object labelled
distance.

\begin{Shaded}
\begin{Highlighting}[]
\NormalTok{distance <-}\StringTok{ }\NormalTok{psych}\OperatorTok{::}\KeywordTok{corr.test}\NormalTok{(mnd_test[,facets]) }\OperatorTok
\StringTok{  }\NormalTok{.}\OperatorTok{$}\NormalTok{r }\OperatorTok
\StringTok{  }\NormalTok{psych}\OperatorTok{::}\KeywordTok{cor2dist}\NormalTok{()}
\end{Highlighting}
\end{Shaded}

So why even bother with transforming correlation coefficients into
distances? The reeason for this can be found in the assumptions about a
geometrical plane. The first assumption is called non-negativty and can
be expressed as: \[d_{ii} = d_{jj} = 0 ≤ d_{ij}\] Simply put on plane
the the distance between any two points \(i\) and \(j\) is greater than
0 or equal to 0 (if \(i = j\)). This presents the first reason why
correlation coefficients can not directly used as input for a MDS,
because they can be negative. The second assumption called symmetry is
self-explanatory: \[d_{ij} = d_{ji}\] For an MDS it is necessary that
the distance between \(i\) and \(j\) is identical to the distance
between \(j\) and \(i\).

Last is the triangle inequality: \[d_{ij} ≤ d_{ik} + d_{kj}\]

This triangle inequality says that going directly from \(i\) to \(j\)
will never be farther than going from \(i\) to \(j\) via an intermediate
point \(k\). If \(k\) happens to be on the way, then the function is an
equality. \#\# Evaluating stress Rather then goodness-of-fit indicators
MDS uses a badness-of-fit indicator, \emph{stress}. Stress is the normed
sum of squares aggregating the representation errors of the model
compared to the undrelying data. In an applied context we rarely examine
the raw stress scores as it is dependent on the scale used. Rather we
use a value to judge badness-of-fit that is called Stress-1 or
\(\sigma_1\) 1 (Kruskal, 1964a). If you are interested, below is the
formula for \(\sigma_1\). \textbf{IT IS CURRENTLY NOT HERE :)}

One important property of \(\sigma_1\) is that missing data is skipped
in the process of summing up the representation errors. This is probably
less a problem if you are working with correlations derived from
underlying data, but can be a problem if you are working with values
obtained from other sources.

From the presented formula we can derive that if we perfectly represent
the underlying data \(\sigma_1\) will be 0 and the greater the deviation
gets the greater \(\sigma_1\) becomes.

\subsubsection{The math}\label{the-math}

\[\sqrt{\sum_{r=1}^{R} (x_ij - x_jr)^2}\] \#\# Metric and Non-metric MDS

\textbf{Include why ratio or ordinal}

\subsection{Interpretation of a MDS
plot}\label{interpretation-of-a-mds-plot}

In their introduction to Multidimensional Scaling Kruskal and Wish
recommend that a MDS plot should be interpreted by applying the
following rule (generalised from their example of Morse code): " Pick
some point which is peripheral, that is, which lies at the outermost
edge of the configuration. Ask yourself what is common to this point and
its nearest neighbors, and how they differ from the points at the
opposite edge of the configuration. Then repeat this process, using
other peripheral points."

In a two dimensional plot it can be beneficial to first examine the x
and y axis. This can yield important insight into the structure of the
points.


\end{document}
